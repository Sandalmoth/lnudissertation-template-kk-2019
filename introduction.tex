\chapter{Introduction}\label{ch:introduction}

\chaptertoc

\noindent \kant[7-8]

Also, use references to entries of several types here: 
\begin{itemize}
\item a book~\cite{Card1999};
\item a book chapter with separate authors~\cite{Fekete2008};
\item a journal article~\cite{VanWijk2006a};
\item a conference paper~\cite{Shneiderman1996};
\item a URL link~\cite{ColorBrewer}; and
\item several references~\cite{Shneiderman1996,Card1999,VanWijk2006a,Fekete2008,ColorBrewer} used in arbitrary order~\cite{VanWijk2006a,ColorBrewer} to check if automatic sorting with \cit{cite} works.
\end{itemize}

\noindent Test a footnote here, too.\footnote{A footnote}. 
Test the quotation environment with a URL link:
\begin{quotation}
\centering
\url{https://lnu.se/}
\end{quotation}

\noindent Test a quotation environment with a text quote:
\begin{quote}
\emph{\cit{A very smart and deep quote \dots}}
\end{quote} 
\noindent Also test another footnote with a URL link.\footnote{\url{https://lnu.se/} (last accessed in February 2019)}


%% The way the section title and mark are defined is to force the correct section mark to appear at the correct page
%% See more details at https://tex.stackexchange.com/a/94901
%% This might not be necessary for all the cases, though
\section[Motivation for Our Problem]{Motivation for Our Research Problem%
\sectionmark{Motivation}%
}\label{sec:intro-motivation}
\sectionmark{Motivation}

\begin{figure}[t!]
\centering
	%\figbox{\includegraphics[width=0.975\linewidth]{images/zzz.pdf}}%
    \figbox{\rule{.1pt}{2cm} \rule[1cm]{2cm}{.1pt} \rule{.1pt}{2cm}}
	\caption[Short caption for example figure]{Long figure caption.
	Explain the contents of the figure here properly.}%
	\label{fig:introduction-example}%
\end{figure} 

\kant[9-15]


\begin{sidewaysfigure}
\centering
	%% One might have to decrease the size of the sideways figure to fit the page margins
	%\figbox{\includegraphics[width=0.93\linewidth]{images/zzz.png}}%
    \figbox{\rule{.1pt}{5cm} \rule[2.5cm]{13cm}{.1pt} \rule{.1pt}{5cm}}
	%% ... and to shrink space to fit the page margins
	%\vspace{-2mm}
	\caption[Sideways figure example]{A long caption for the sideways figure here.}%
	\label{fig:introduction-example-sideways}%
	%\vspace{-2mm}
\end{sidewaysfigure}

\begin{figure}[t!]
%\centering
	\begin{subfigure}[t]{0.477\linewidth}
			\centering
			%\figbox{\includegraphics[width=\linewidth]{images/zzz.png}}%
			\figbox{\rule{.1pt}{2cm} \rule[1cm]{2cm}{.1pt} \rule{.1pt}{2cm}}
			\caption{}%
	\end{subfigure}%
	\hspace{3pt}
	\begin{subfigure}[t]{0.4754\linewidth}
			\centering
			%\figbox{\includegraphics[width=\linewidth]{images/zzz.png}}%
			\figbox{\rule{.1pt}{2cm} \rule[1cm]{2cm}{.1pt} \rule{.1pt}{2cm}}
			\caption{}%
	\end{subfigure}%
	%\vspace{-2mm}
	\caption[Short caption for a figure with subfigures]{A figure with subfigures with long captions. 
	(a) A dedicated caption could be provided directly in the subfigure code, but a long caption text arguably suits this area here better. 
    (b) The same applies to the second subfigure.
	}%
	\label{fig:intro-subfigures}%
\end{figure} 


\subsection{Subsection Here}\label{subsec:intro-subsection}

\begin{table}[t!]
\caption[Short caption for a table]{An arbitrary table}%
\label{tab:introduction-table}%
%\renewcommand{\arraystretch}{1.2}
\begin{minipage}{0.975\textwidth}
\centering
\begin{tabular}{lll}
%% The header
\toprule 
\parbox{0.26\textwidth}{\centering\textbf{Title}}
&
\parbox{0.32\textwidth}{\centering\textbf{Description}}
&
\parbox{0.33\textwidth}{\centering\textbf{Examples}} \\ 
\midrule
%% The body
\parbox[c][9mm]{0.26\textwidth}{Foo, bar,\\[-2pt]and baz}
&
\parbox[c]{0.32\textwidth}{ %
\scriptsize
Expression of foo and bar}
&
\parbox[c]{0.33\textwidth}{ %
\vspace{2pt}
\scriptsize
\emph{Example1}; \emph{Example2}
\vspace{2pt}} \\
\parbox[c][9mm]{0.26\textwidth}{Foo, bar,\\[-2pt]and baz}
&
\parbox[c]{0.32\textwidth}{ %
\scriptsize
Expression of foo and bar}
&
\parbox[c]{0.33\textwidth}{ %
\vspace{2pt}
\scriptsize
\emph{Example1}; \emph{Example2}
\vspace{2pt}} \\
%% The footer
\bottomrule
\end{tabular}% 
\end{minipage}%

\bigskip
\raggedright
\footnotesize{\emph{Note:} Adjust the column widths appropriately. 
And this is the area for long table caption notes, by the way. 
}
\end{table} 


\kant[28-31]


\begin{sidewaystable}
%\vspace{-4mm}
\caption[Short caption for a complex sideways table]{A complex sideways table consisting of several parts}
\label{tab:introduction-sideways-table}
\centering
\setlength{\tabcolsep}{2pt}
\renewcommand{\arraystretch}{1.2}
\setlength\doublerulesep{2mm} 
\footnotesize
\begin{minipage}[t]{0.27\textwidth}
\begin{tabular}[t]{lr}
\toprule
\textbf{Group header} & \textbf{100}\\ 
\midrule
Foo & 75\\ 
Bar & 20\\
Baz & 5\\
\addlinespace
\addlinespace
\textbf{Group header} & \textbf{100}\\ 
\midrule
Foo & 75\\ 
Bar & 20\\
Baz & 5\\
\addlinespace
\addlinespace
\textbf{Group header} & \textbf{100}\\ 
\midrule
Foo & 75\\ 
Bar & 20\\
Baz & 5\\
\bottomrule
\end{tabular} 
\end{minipage}
\hspace{1mm}
\begin{minipage}[t]{0.445\textwidth}
\begin{tabular}[t]{lr}
\toprule
\textbf{Group header} & \textbf{100}\\ 
\midrule
Foooooooooooooooooo & 75\\ 
Barrrrrrrrrrrrrrrrrrrrrrrrrrr & 20\\
Bazzzzzzzzzzzzzzzzzzz & 5\\
\addlinespace
\addlinespace
\textbf{Group header} & \textbf{100}\\ 
\midrule
Foooooooooooooooooo & 75\\ 
Barrrrrrrrrrrrrrrrrrrrrrrrrrr & 20\\
Bazzzzzzzzzzzzzzzzzzz & 5\\
\bottomrule
\end{tabular} 
\end{minipage}
\hspace{1mm}
\begin{minipage}[t]{0.22\textwidth}
\begin{tabular}[t]{lr}
\toprule
\textbf{Group header} & \textbf{100}\\ 
\midrule
Foo & 75\\ 
Bar & 20\\
Baz & 5\\
\addlinespace
\addlinespace
\textbf{Group header} & \textbf{100}\\ 
\midrule
Foo & 75\\ 
Bar & 20\\
Baz & 5\\
\addlinespace
\addlinespace
\textbf{Group header} & \textbf{100}\\ 
\midrule
Foo & 75\\ 
Bar & 20\\
Baz & 5\\
\bottomrule
\end{tabular} 
\end{minipage}
%\vspace{-5mm}

\bigskip
\raggedright
\footnotesize{\emph{Note:} Adjust the minipage widths appropriately.
}
\end{sidewaystable}

\subsubsection{Subsubsection Here}\label{subsubsec:intro-subsubsection}
\kant[32]

\paragraph{A Named Paragraph}
\kant[33]



